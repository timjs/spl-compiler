\input beamerlayout

\lstnewenvironment{code}
  {\lstset{language=haskell,
           keywords={Reader,ReaderT,Writer,WriterT,do,data,type,ask,tell},
           otherkeywords={=,::,->,<-,\$,=~,<\&>}}} 
  {}

\begin{document}

\subtitle {Compiler Construction}
\title    {An analyser for \SPL}
\author   [Tim Steenvoorden]
          {Tim Steenvoorden\\
           \medskip
           {\small \url{tim.steenvoorden@gmail.com}}}%@
\institute{Radboud University Nijmegen}
\date     {25 March 2013}

\begin{frame}
  \titlepage
\end{frame}

%\begin{frame}{Outline}
  %\tableofcontents
%\end{frame}

\section{Introduction}
\subsection*{}

\begin{frame}{Goals}
  
  Learn more about monads and monad transformers!

  %\pause
  %We are going to use them today!

  \pause
  \bigskip
  When checking statements and matching types we need to:

  \begin{enumerate}
    \item Read type information from an environment.
    \item Write errors to a log.
  \end{enumerate}

\end{frame}

\section{Reading}
\subsection*{}

\begin{frame}[fragile]{Environments}

  Map names (variables, functions) to types:

    \begin{code}
      type Environment = Map Name Type
    \end{code}
    
  \pause
  
  \begin{block}{}
    \begin{code}
      check :: Environment -> Statement -> Bool !\pause!
      check env (Assign n e) = match env e (lookup n env)
      ...
    \end{code}
  \end{block}
  
  \pause

  \begin{itemize}
    \item Need to carry around an extra parameter \frownie
  \end{itemize}

\end{frame}

\begin{frame}[fragile]{Reading Environments}
  
  \begin{block}{}
    \begin{code}
      check :: Environment -> Statement -> Bool
      check env (Assign n e) = match env e (lookup n env)
      ...
    \end{code}
  \end{block}
  
  Meet the Reader monad!

  \begin{block}{Reader}
    \begin{code}
      check :: Statement -> Reader Environment Bool
      check (Assign n e) = do
        env <- ask            -- We have to ask for it
        e =~ lookup n env     -- Use nice operator !\pause!
      check (If c ts fs) = c =~ BOOL <&> check ts <&> check fs
      ...
    \end{code}
  \end{block}

\end{frame}

\section{Writing}
\subsection*{}

\begin{frame}[fragile]{Errors}

  Abstract error type:

    \begin{code}
      data Error = TypeMismatch Expression Type | ...
    \end{code}
  
  \pause

  \begin{block}{}
    \begin{code}
      (=~) :: Expression -> Type -> (Bool,[Error])
    \end{code}
  \end{block}

  \pause

  \begin{itemize}
    \item Have to extract our |Bool| \frownie
    \item Need to paste all errors together \frownie
  \end{itemize}

  \pause
  Meet the Writer monad!

  \begin{block}{Writer}
    \begin{code}
      (=~) :: Expression -> Type -> Writer [Error] Bool
    \end{code}
  \end{block}

\end{frame}

\section{Analysing}
\subsection*{}

\begin{frame}[fragile]{Analysing a Program}
  
  \begin{description}
    \item[Reader] implicitly passes around our environment
    \item[Writer] implicitly collects all our errors
  \end{description}

  \pause
  \bigskip

  Stack them:

  \begin{block}{}
    \begin{code}
      type Analyser = ReaderT Environment (Writer Error)
    \end{code}
  \end{block}

  using monad transformers!

\end{frame}

\section{Demo}
\subsection*{}

\begin{frame}{Demo}
  Code on GitHub: \url{https://github.com/timsteenvoorden/spl-compiler}

  \bigskip

  See how it works!
\end{frame}

\appendix

\begin{frame}[fragile]{Errors}

  Abstract error type:

    \begin{code}
      data Error = TypeMismatch Expression Type | ...
    \end{code}
  
  \pause

  \begin{block}{}
    \begin{code}
      (=~) :: Expression -> Type -> (Bool,[Error]) !\pause!
      Boolean _ =~ BOOL   = (True, [])
      Nil       =~ LIST _ = (True, [])
      ... !\pause!
      e         =~ t      = (False, [TypeMismatch e t])
    \end{code}
  \end{block}

  \pause

  \begin{itemize}
    \item Have to extract our |Bool| \frownie
    \item Need to paste all errors together \frownie
  \end{itemize}

\end{frame}

\begin{frame}[fragile]{Writing Errors}

  \begin{block}{}
    \begin{code}
      (=~) :: Expression -> Type -> (Bool,[Error])
      Boolean _ =~ BOOL   = (True, [])
      Nil       =~ LIST _ = (True, [])
      ...
      e         =~ t      = (False, [TypeMismatch e t])
    \end{code}
  \end{block}

  Meet the Writer monad!

  \begin{block}{Writer}
    \begin{code}
      (=~) :: Expression -> Type -> Writer [Error] Bool
      Boolean _ =~ BOOL   = return True
      Nil       =~ LIST _ = return True
      e         =~ t      = do
        tell [TypeMismatch e t] -- We have to tell it
        return False
    \end{code}
  \end{block}

\end{frame}


\end{document}



  \begin{enumerate}
    \item Use appropriate data types

          (instead of ``lists for everything'')

    \bigskip

    \item Use sophisticated control structures

          (instead of ``passing all things around'')
  \end{enumerate}

