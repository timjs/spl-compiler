\input beamerlayout

\lstnewenvironment{HASKELL}
  {\lstset{language=haskell}}
  {}

\begin{document}

\subtitle {Compiler Construction}
\title    {A parser for \SPL}
\author   [Tim Steenvoorden]
          {Tim Steenvoorden\\
           \medskip
           {\small \url{tim.steenvoorden@gmail.com}}}%@
\institute{Radboud University Nijmegen}
\date     {18 March 2013}

\begin{frame}
  \titlepage
\end{frame}

\begin{frame}{Outline}
  \tableofcontents
\end{frame}

\section{Tools}
\subsection*{}

\begin{frame}{Language}
  
  Choosing a language was easy
  
  I like Haskell for:

  \begin{itemize}
    \item Its rigid type system
    \item Its nice programing paradigm
    \item Its great libraries
    \item Its large community
  \end{itemize}

  But most importantly:

  \begin{itemize}
    \item Understand more about functors, monads, transformers etc.
  \end{itemize}

\end{frame}

\begin{frame}{Libraries}

  Most time consuming:

  \begin{itemize}
    \item Looking for the right library and settle down

    (Parsec, Attoparse, Polyparse, UU ParseLib, \\
     Pretty, WL-PPrint, \ldots)
  \end{itemize}

  \pause
  But finally:

  \begin{description}
    \item[Parser]  Daan Leijen's parser combinators \emph{Parsec}\\
                   recursive descent, look ahead 1 ($LL(1)$)
    \item[Lexer]   Some simple Parsec functions,\\
                   generates tokens on the fly
    \item[Printer] Philip Wadler's \emph{Prettier Printer}\\
                   including some syntax highlighting!
  \end{description}

\end{frame}

\section{Parsing}
\subsection*{}

\begin{frame}[fragile]{Parsing (Type,Type)}
    \begin{HASKELL}
      data Type = PAIR Type Type | ...
    \end{HASKELL}

  \begin{block}{Monadic}
  \begin{twocolumns}
    \begin{HASKELL}
      pair :: Parser Type
      pair = do paren
                x <- atype
                comma
                y <- atype
                paren
                return (PAIR x y)
    \end{HASKELL}
    \nextcolumn
    \pause
    \begin{HASKELL}
      pair :: Parser Type
      pair = paren >>
             atype >>= \x ->
             comma
             atype >>= \y ->
             paren
             return $ PAIR x y
    \end{HASKELL}
  \end{twocolumns}
  \end{block}

  \pause
  \begin{block}{Applicative}
    \begin{HASKELL}
      PAIR <$> (paren *> atype)
           <*> (comma *> atype <* paren)
    \end{HASKELL}
  \end{block}

\end{frame}

\begin{frame}[fragile]{Applicative Approach}

  \begin{block}{Classes and Functions}
    %\begin{HASKELL}
    %   $  ::   (a -> b) ->   a ->   b
    %  <$> ::   (a -> b) -> f a -> f b
    %  <$  ::         a  -> f b -> f a
    %  <*> :: f (a -> b) -> f a -> f b
    %  <*  :: f a        -> f b -> f a
    %   *> :: f a        -> f b -> f b
    %\end{HASKELL}
    \begin{description}
      \item[Functor]     |<$>| (map), |<$|
      \item[Applicative] |<*>| (ap), |<*|, |*>| (|>>|)
    \end{description}
  \end{block}

  \pause
    \begin{HASKELL}
      atype :: Parser Type
      atype =   VOID <$  reserved "Void"
            <|> INT  <$  reserved "Int"
            <|> BOOL <$  reserved "Bool"
            <|> PAIR <$> (paren *> atype)
                     <*> (comma *> atype <* paren)
            <|> LIST <$> brackets atype
            <|> TYPE <$> identifier
            <?> "type annotation"
    \end{HASKELL}

  \pause

  Grammar stays almost the same!

\end{frame}

\section{Experiments}
\subsection*{}

\begin{frame}{Experiments}
  
  Other experiments I tried out or wanted to try out

  \begin{itemize}
    \item Use \verb|Text| instead of \verb|String|

    \verb|String|s are based on \verb|List|s,
    \verb|Text| is more modern and space efficient\\
    $\implies$ not all libraries support it, we have to convert things

    \pause
    \item Use a \verb|Writer| monad to collect error messages

    \verb|Writer| can keep track of ``log messages'',
    \verb|Parsec| is a monad transformer, so in principle this should work
  \end{itemize}

\end{frame}

\section{Demo}
\subsection*{}

\begin{frame}{Demo}
  \ldots
\end{frame}

\end{document}

% vim: ft=latex
